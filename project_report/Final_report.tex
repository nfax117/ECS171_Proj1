\documentclass{article}
\usepackage[utf8]{inputenc}
\usepackage{amssymb,amsmath,titling}
\usepackage{amssymb,mathtools,amsthm,amsmath}
\usepackage{float}
\usepackage{indentfirst}
\usepackage{listings}
\usepackage{xcolor}

\definecolor{codegreen}{rgb}{0,0.6,0}
\definecolor{codegray}{rgb}{0.5,0.5,0.5}
\definecolor{codepurple}{rgb}{0.58,0,0.82}
\definecolor{backcolour}{rgb}{0.95,0.95,0.92}

\lstdefinestyle{mystyle}{
    backgroundcolor=\color{backcolour},   
    commentstyle=\color{codegreen},
    keywordstyle=\color{magenta},
    numberstyle=\tiny\color{codegray},
    stringstyle=\color{codepurple},
    basicstyle=\ttfamily\footnotesize,
    breakatwhitespace=false,         
    breaklines=true,                 
    captionpos=b,                    
    keepspaces=true,                 
    numbers=left,                    
    numbersep=5pt,                  
    showspaces=false,                
    showstringspaces=false,
    showtabs=false,                  
    tabsize=2
}

\lstset{style=mystyle}
\title{ECS 171 Final Project Report}
\author{
  Sergio Santoyo\\
  \and
  Yuan Chang\\
  \and
  Cesar Guzman Avina\\
  \and
  Will Colbert\\
  \and
  Nathaniel Faxon\\
  \and
  Kanchan Kaur\\
  \and
  Parminder Singh\\
}
\date{May 2022}

\begin{document}

\maketitle
\section{Introduction \& Background}
The stock market has evolved into a major component of our lives, as well as a reflection of the national and global economy. A stock represents ownership, or equity stake, in the company, and has served as a way for companies to raise capital. Additionally, the stock market enables individuals to grow their wealth, while holding companies accountable and keeping an eye on corporate regulation. The stock market impacts the overall economy, and therefore, impacts everyone, including non-investors. Because stock markets span all industries and sectors, it is a key factor in determining the state and cycle of the economy.

When the stock market crashes, there is a sudden decline of stock prices, which results in a loss of wealth, and are further driven by other economic factors and current events. A crash is propelled with investors selling in panic, dropping prices even more. This provides an opportunity for investors, as they can potentially buy stocks back at a much lower price.

Machine learning can be used to identify stock market and financial trends, typically done using the LSTM (Long Short-term Memory) model, support vector machines (SVM), artificial neural networks (ANN), and back propagation neural networks (BPNN). While machine learning can be used to predict stock prices and market crashes, this is a challenging problem. The stock market can be volatile, influenced by many factors, ranging from politics, unexpected events, psychological factors, a company’s performance, and so on. This dynamic nature of the stock market makes it difficult to accurately predict crashes and prices.

If a market prediction model is built to successfully predict a stock market crash, it can be beneficial to investors, while also giving a warning to companies and the general public. It can also be used in various fields/sectors. Here are a some examples: 

\newpage
	\begin{itemize}
		\item Communication and Media
			\begin{itemize}
				\item Machine learning can process content on social media platforms from high-power individuals in the stock market, and predict the market in multiple scenarios.   
			\end{itemize}
		\item Finance
			\begin{itemize}
				\item The finance sector is one of the most hit sectors during a crash. Predicting a crash could allow financial institutions to prepare for the consequences, like anticipating layoffs.
			\end{itemize}
		\item Leisure and Hospitality 
			\begin{itemize}
				\item Leisure and hospitality companies face significant economic struggles after a crash, making it difficult for smaller companies to remain in business. Predicting a crash would allow these companies to prepare for such financial hardship, while also taking some action to keep customers, like offering discounted services that can be bought ahead of time, so customers will still participate after the crash. 
			\end{itemize}
	\end{itemize}
 
\section{Literature Review} 
\section{Dataset Description and Exploratory Data Analysis} 
\section{Proposed Methodology} 
\section{Experimental Results} 
\section{Conclusion and Evaluation} 
 
\end{document}








